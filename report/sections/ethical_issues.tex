\section{Ethical Issues Identified - StudyBuddy}

\subsection{Introduction}
StudyBuddy is a collaborative platform intended to support university students by enabling peer-to-peer study matching, scheduling, communication, and resource sharing. Because the system may process personal data, enable interaction between users, and potentially touch on wellbeing-related experiences, its design must be ethically grounded and aligned with professional standards for responsible computing.

\subsection{Data Privacy and Data Protection (UK GDPR)}
StudyBuddy may handle direct identifiers (name, email, profile photo) and indirect identifiers (timetable patterns, modules, study preferences). StudyBuddy should treat student data as sensitive and minimise collection to what is essential for the core service.

\textbf{Ethical obligations and mitigations:}
\begin{itemize}
    \item \textbf{Data minimisation:} collect only what is necessary.
    \item \textbf{Purpose limitation:} use data only for stated purposes.
    \item \textbf{Storage limitation:} define retention periods and deletion processes.
    \item \textbf{Integrity and confidentiality:} implement secure storage, transport security, and access control.
\end{itemize}

\subsection{Informed Consent and Transparency}
Ethical systems require that users understand what they agree to. StudyBuddy must provide clear information about what data is collected, how matching works, how long data is kept, who can see what, and how users can delete accounts and data.

\subsection{Security, Confidentiality, and Secure Engineering}
Security is not merely technical; it is an ethical duty because breaches can harm users and erode trust. Professional standards emphasise protecting confidentiality and preventing foreseeable misuse.

\subsection{User Safety, Harassment, and Moderation}
Any platform enabling messaging, matching, or user-to-user interaction carries risk of harassment, bullying, and unwanted contact. StudyBuddy should incorporate reporting mechanisms, moderation procedures, and basic safety features like blocking or muting.

\subsection{Inclusivity and Accessibility}
Inclusivity requires that StudyBuddy be usable for students with diverse needs, including those using assistive technologies. Following WCAG guidance helps reduce exclusion and supports fairness.

\subsection{Bias, Fairness, and Matching Algorithms}
Even simple matching can introduce bias. StudyBuddy should define fair matching criteria (subject, level, availability) rather than sensitive personal traits and provide transparency about how matches are generated.

\subsection{Wellbeing Features and Mental Health Sensitivity}
If StudyBuddy includes wellbeing components, it must not imply clinical authority. Safeguards include using careful language, avoiding diagnosing users, and signposting to official university or NHS support resources.

\subsection{Academic Integrity and Intellectual Property}
As a study support platform, StudyBuddy could be misused to facilitate plagiarism or answer-sharing. Mitigation expectations include clear platform rules promoting collaboration, content guidelines, and reporting mechanisms for academic misconduct.
