\section{Ethical Issues Identified – StudyBuddy}

\subsection{Introduction}
StudyBuddy must be ethically grounded and aligned with professional standards. Ethical risks emerge from poor requirements, ambiguous consent, weak security, and exclusionary interface decisions.

\subsection{Data Privacy and Data Protection (UK GDPR)}
StudyBuddy handles direct identifiers (name, email) and indirect identifiers (timetable patterns, modules). Mitigation includes data minimisation, purpose limitation, storage limitation, and integrity/confidentiality.

\subsection{Informed Consent and Transparency}
Users must understand what data is collected, how matching works, retention periods, and how to delete accounts.

\subsection{Security, Confidentiality, and Secure Engineering}
Security is an ethical duty. Mitigation include strong hashing, parameterized queries, and environment variable management.

\subsection{User Safety, Harassment, and Moderation}
Risks include harassment and unwanted contact. The platform incorporates a Code of Conduct, reporting mechanisms, and block/mute features.

\subsection{Inclusivity and Accessibility}
Accessibility is a core expectation. Key concerns include readability, keyboard navigation, and responsive design following WCAG guidance.

\subsection{Bias, Fairness, and Matching Algorithms}
Matching can introduce bias. We define fair matching criteria (subject, level, availability) and provide transparency.

\subsection{Wellbeing Features and Mental Health Sensitivity}
Wellbeing features must not imply clinical authority. Safeguards include careful language and signposting to official resources.

\subsection{Academic Integrity and Intellectual Property}
Mitigations include clear rules against plagiarism and academic misconduct, and reporting mechanisms for direct answer exchanges.
